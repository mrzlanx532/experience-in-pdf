\documentclass[a4paper, 12px]{article}

\usepackage{cmap}% поиск в PDF
\usepackage[T2A]{fontenc} % кодировка
\usepackage[utf8]{inputenc} % кодировка исходного текста
\usepackage[english,russian]{babel} % локализация и переносы

\usepackage{xcolor}

\usepackage{hyperref}



\usepackage{graphicx}

\graphicspath{{./images/}}


\author{ \href{https://github.com/mrzlanx532}{mrzlanx532}}
\title{All experience}
\date{\today}

\begin{document}
	
\maketitle

\tableofcontents{}
\clearpage 

\section{Git commands}

\subsection{Git common's commands}


\textbf{git help <команда>}\\открывает документацию по команде\\\\
\textbf{git init}\\инициализация git-репозитория


\subsection{Git config}

\textbf{git config -\--global core.editor *редактор кода*}\\ установка стандартного редактора кода\\\\
\textbf{git config -\--list}\\список всех настроек локально (или глобально -\--global)\\\\
\textbf{git config -\--global alias.co checkout}\\создание краткого алияса (команда checkout будет выполняться через команду "co")\\\\
\textbf{git config -\--global alias.co 'checkout -b'}\\создание краткого алияса "co"\ c параметрами\\\\
\textbf{git config core.fileMode false}\\выключить отслеживания за ПРАВАМИ ФАЙЛА

\subsection{Git status}

\textbf{git status}\\проверка на существование измененных файлов\\\\
\textbf{git status -s (-\--short)}\\краткий вывод состояния файлов

\subsection{Git add}

\textbf{git add *}\\Добавить все файлы в индексацию\\\\
\textbf{git add <file>}\\Добавить <file> в индексацию

\subsection{Git clone}

\textbf{git clone https://github.com/libgit2/libgit2.git}
\\клонирование репозитория\\\\
\textbf{git clone <url>}\\клонировать репозиторий\\\\
\textbf{git clone -o <локальное\_название\_источника>}\\склонировать репозиторий, и дать источнику имя <локальное\_название\_источника>

\subsection{Git diff}

\textbf{git diff}\\посмотреть изменение файлов в консоли до индексации\\\\
\textbf{git diff -\--cached}\\покажет проиндексированные изменения\\\\
\textbf{git diff -\--staged}\\что из проиндексированного войдет в следующий коммит, воспользуйтесь командой

\subsection{Git log}


\textbf{git log}\\показать все последние коммиты\\\\
\textbf{git log -p}\\показать все последние коммиты с изменениями\\\\
\textbf{git log -p -1}\\показать 1 последний коммит с изменениями\\\\
\textbf{git log -\--stat}\\показать все коммиты с измененными файлами кратко\\\\
\textbf{git log -\--pretty=oneline}\\показ комитов с описанием\\\\
\textbf{gitlog -\--graph}\\удобное представление логов\\\\
\textbf{git log -\--relative-date}\\относительная дата коммита (35 минут назад)

\subsection{Git branch}

\textbf{git branch -d <имя\_ветки>}\\Удалить локальную ветку\\\\
\textbf{git branch <имя\_локальной\_ветки>}\\создать локальную ветку\\\\
\textbf{git branch -d <имя\_ветки>}\\удалить локальную ветку\\\\
\textbf{git branch -u <удаленная\_ветка>}\\начать отслеживать удаленную ветку с текущей\\\\
\textbf{git branch -vv}\\показать информацию о ветках и за чем они следят

\subsection{Git remote}

\textbf{git remote -v}\\посмотреть подключенный удаленный источник\\\\
\textbf{git remote add <локальное\_название\_источника> <url>}\\добавить новый источник\\\\
\textbf{git remote rename <старое\_имя\_уд.\_реп> <новое\_имя\_уд.\_реп>}\\переименовать удаленный источник\\\\
\textbf{git remote rm <имя\_уд.\_реп>}\\удалить ссылку на удаленный реп

\subsection{Git stash}

\textbf{git stash}\\убрать в стэш изменения\\\\\\
\textbf{git stash list}\\показать список изменений в стэшэ\\\\
\textbf{git stash pop}\\вытащить из стэша (и удалить из стэка стэша) последние изменения\\\\
\textbf{git stash apply}\\вытащить последние изменения (но не удалять из стэка стэша) спрятаннные в стэш\\\\
\textbf{git stash list | pop stash@\{0\}}\\вытащить по порядковому номеру изменения из стэша\\\\
\textbf{git stash drop stash@\{0\}}\\удалить из стэка стэша изменения у последнего коммита\\\\
\textbf{git stash -\--keep-index}\\убрать в стэш только непроиндексированные файлы (красные)\\\\
\textbf{git stash -u}\\скрыть файлы в стэш, включая те, которые не находятся в индексации\\\\
\textbf{git stash show stash@\{0\}}\\
показать файлы в последнем стеше

\subsection{Git merge\textbackslash mergetool}

\textbf{git merge <название\_ветки>}\\слить текущую ветку с <название\_ветки>\\\\
\textbf{git merge -\--abort}\\отменить слияние (например при конфликте)\\\\
\textbf{git merge -\--no-commit -\--squash featureB}\\-\--squash - сливает изменения в один коммит, не являющийся коммитом слияния\\
-\--no-commit - отменяет автоматическую запись коммита\\\\
\textbf{git mergetool}\\открыть редактор по умолчанию для разрешения конфликта\\\\
\textbf{git mergetool -\--tool=<название\_утилиты>}\\открыть редактор <редактор> для разрешения конфлита

\subsection{Git rm\textbackslash mv}

\textbf{git rm <файл>}\\удаляет файл и вносит в индексацию `удаление` этого файла\\\\
\textbf{git rm -\--cached <файл>}\\удаляет из индекса, но физически не удаляет его\\\\\\
\textbf{git mv <изначальный\_файл> <новое\_название\_файла>}\\
переименование с индексацией

\subsection{Git checkout}

\textbf{git checkout <имя\_ветки>}\\переключится на ветку\\\\
\textbf{git checkout -b <имя\_новой\_ветки>}\\создать ветку и переключится на нее\\\\
\textbf{git checkout -\-- <файл>}\\Вернуть <файл> к дефолту. (Убрать внесенные изменения)

\subsection{Git others}

\textbf{git push origin -\--delete <имя\_ветки>}\\Удалить ветку из удаленного репозитория\\\\
\textbf{git commit -a}\\пропуск индексирования и сразу коммит\\\\
\textbf{git commit -\--amend}\\слияние текущего коммита с предыдущим\\\\
\textbf{git reset HEAD <имя\_файла>}\\убрать из индексирования файл\\\\
\textbf{git fetch <локальное\_название\_источника>}\\скачать без слияния\\\\ 
\textbf{git pull <локальное\_название\_источника>}\\скачать со слиянием\\\\
\textbf{git diff master}\\смотрим изменения\\\\
\textbf{git push origin -\--delete <имя\_ветки>}\\удалить ветку с удаленного репозитория\\\\
\textbf{git rebase <имя\_ветки>}\\тоже самое что и git merge, но только история коммитов отображается по другому\\\\
\textbf{git request-pull origin/master myfork}\\запрос на включение коммитов в удаленный реп "origin/master" с ветки "myfork"\\\\
\textbf{git revert -m 1 HEAD}\\отменить последний коммит с фиксацией в истории\\\\
\textbf{git blame -L 12,22 simplegit.rb}\\посмотреть кто сделал изменения в файле "simplegit.rb" с 12 строки до 22 строки

\subsection{Git ignore}

Пример файла gitignore\\
\noindent\rule{\textwidth}{1pt}
\# - комментарий\\
*.a - пропускать файлы, заканчивающиеся на ".a"\\
!lib.a \# - но отслеживать файлы "lib.a", несмотря на пропуск файлов на ".a"\\
/TODO - игнорировать только корневой файл "TODO", а не файлы вида "subdir/TODO"\\
build/ - игнорировать все файлы в папке "build/"\\
doc/*.txt - игнорировать "doc/notes.txt", но не "doc/server/arch.txt"\\
\noindent\rule{\textwidth}{1pt}
\href{https://github.com/github/gitignore}{https://github.com/github/gitignore} - здесь можно посмотреть различные варианты настройки файла .gitignore

\section{PostgreSQL}

Установка клиента и службы: \\

\textbf{sudo apt-get install postgresql postgresql-contrib}\\\\
Авторизоваться под пользователем postgres: \\

\textbf{sudo -i -u <postgres>} \\\\
Войти в клиент-консоль postgres: \\

\textbf{sudo -i -u <postgres> psql}\\\\
Создание пользователя интерактивно \\

\textbf{createuser -\--interactive}  \\\\
Создание БД\\

\textbf{createdb <db\_name>}\\\\
Меняем пароль авторизации в клиент-консоли: \\

\textbf{ALTER USER <mrzlanx532> with PASSWORD 'password';}\\\\
Установка Postgres-PDO для PHP7.*\\

\textbf{sudo apt-get install 7.*-pgsql}\\\\
\textbf{Laravel .env}\\\\
\textit{	
DB\_CONNECTION=pgsql\\
DB\_HOST=localhost\\
DB\_PORT=5432\\
DB\_DATABASE=db\\
DB\_USERNAME=username\\
DB\_PASSWORD=password\\}

\section{Regular Expresions (RegExp)}

\textbf{Паттерн | Пример} \\\\
$\hat{}$ = не \\
duck = "duck"\\
p|tyre = "p" или "tyre"\\
(p|t)yre = "pyre" или "tyre"\\
$[$dtl$]$uck = "duck", "tuck", "luck" \\
$[$ $\hat{}$ t]uck = все буквы вначале, кроме t. "tuck" нельзя, "buck" можно \\
$[$a-z$]$ = диапазон от a до z \\
$[$A-Z$]$ = диапазон от A до Z \\
$[$0-9$]$ = диапазон от 0 до 9 \\
$[$0-9$]$+ = бесконечное количество 8784655351518484213548 \\
$[$0-9$]$\{11\} = 11 символов 89645327312 \\
$[$0-9$]$\{5,7\} = 5-7 символов 89645327312 \\
$[$0-9$]$\{5, \} = минимум 5 символов 89645327312 \\\\
\textbf{Мета символы}\\\\
$\backslash$d = [0-9] \\
$\backslash$w = [a-zA-Z0-9\_] \\
$\backslash$s = пробелы, табы, отступы \\
$\backslash$t = любой символ табуляции \\\\
\textit{Пример}\\\\
$\backslash$d\{3\}$\backslash$s$\backslash$w\{5\} = 123 hello \\\\
\textbf{Особые символы} \\
+ - совпадение встречается один или более раз \\
? - символ либо есть, либо нет \\
. - абсолютно любой символ \\
* - совпадение встречается ноль или более раз \\\\
\textit{Пример}\\\\
hello? = hell, hello \\\\
\textbf{Начало и конец шаблона} \\\\
$\hat{}$ [a-z]\{5\} = davidgtropgrpk (подойдет только david без продолжения) $\hat{}$ это начало \\
$[$a-z$]$\{5\}\$ = fefefewwwww (подойдет только wwwww без начала) \$ это конец \\
$\hat{}$ $[$a-z$]$\{5\}\$ = david (нельзя ни в начале ничего добавить, ни в конце) \\

\section{Python}

\subsection{Установка virtualenv и зависимостей Python} 
1. \textbf{py -m pip install virtualenv} - установка \textbf{virtualenv} в \textbf{pip-packages} \\
2. \textbf{mkdir python\_env \&\& cd python\_env} - создаем папку под окружения python и переходим в нее \\
3. \textbf{virtualenv <название\_проекта>} (e.g. "virtualenv django\_proj") - создаем переменную окружения под название "django\_proj"\\
4. \textbf{source <название\_проекта>/bin/activate} - заходим включаем окружение\\
5. \textbf{pip install <название\_модуля>} (e.g. \textbf{"pip install Django"}) - устанавливаем модули в окружение [OPTIONAL]\\
6. \textbf{pip freeze -\--local > req.txt} - сохраняем зависимости локального окружения в файл\\
7. \textbf{deactivate} - выйти из окружения
\subsection{Деплой проекта питона с зависимостями из файла используя virtualenv}
1. \textbf{virtualenv -p /path\_to\_python\_exe <название\_переменной\_окружения>} (e.g \textbf{virtualenv -p /usr/bin/python3.8 py3.8-django}) - создаем переменную окружения из зависимостей, используя версию питона \\
2. \textbf{source py3.8-django/bin/activate} - заходим в п.о. \\
3. \textbf{pip install -r req.txt} - ставим зависимости в переменную окружения
\subsection{Установка PyQt5-designer на Linux} 
1. sudo apt-get install python3-pyqt5 \\
2. sudo apt-get install python3-pyqt5.qtsql\\
3. sudo apt-get install qttools5-dev-tools

\subsection{Запуск PyQt5-designer как приложение}
1. Создайте командой файл на рабочем столе \textbf{'touch "qt-designer.desktop"'} \\
2. Откройте файл в редакторе и внесите следующее содержимое:\\\\
\textit{\#!/usr/bin/env xdg-open}\\
\textit{[Desktop Entry]}\\
\textit{Name=Qt5 Designer}\\
\textit{Icon=/usr/share/applications/qt5-designer\_.png}\\
\textit{Exec=/usr/lib/x86\_64-linux-gnu/qt5/bin/designer}\\
\textit{Type=Application}\\
\textit{Categories=Application}\\
\textit{Terminal=false}\\
\textit{StartupNotify=true}\\
\textit{Actions=NewWindow}\\
\textit{Name[en\_US]=Qt5 Designer}\\\\
\textit{[Desktop Action NewWindow]}\\
\textit{Name=Open a New Window}\\
\textit{Exec=/usr/lib/x86\_64-linux-gnu/qt5/bin/designer}\\\\
Если необходима иконка приложения, добавьте созданную иконку в "/usr/share/applications"


\end{document}